\documentclass[a4paper, 12pt, titlepage]{article}

% Including needed packages
\usepackage[margin=2cm]{geometry}
\usepackage{amsmath}

\title
{{\em INF 551 - Computer-aided reasoning}\\
Mini-project - Encoding funny games into SAT\\
{\bf Report}}
\author{ROHRMANN Till \and GROSSHANS Nathan}
\date{}

\begin{document}

\maketitle


\section{Introduction}
To complete.


\section{Skyscraper}
The skyscraper game is a little logical game that works like the Sudoku game.
Having a square grid of a fixed size, your goal is to place skyscrapers of
different heights (from 1 to the size of the grid) in the squares of the grid,
such that a skyscraper of each height appears once and only once on each row and
each column, and that the visibility constraints are respected. The visibility
constraints are given by numbers, one on each side of a row or column,
indicating how many buildings must be seen when looking from the given side
towards the skyscraper row or column. There can also be no constraint at all for
a given row or column side.

\subsection{The encoding}
To complete.

\subsection{How to use the program}
To complete.


\section{Hex-a-Hop}
Hex-a-Hop is a puzzle-based game with hundred levels where the player jumps from
one hexagonal tile to another. The goal, for each level, is to destroy all the
green tiles by visiting them once, without getting trapped or falling into the
water. The levels gradually become more and more difficult, unveiling more and
more new tile types having different behaviours (trampolines, high tiles...).

\subsection{The encoding}

\subsubsection{Representing a level}
A level can be represented by a grid (field) of tiles of different types. As
these tiles are hexagonal, the representation is a bit more complicated as if
they where just squares. In fact, as two successive rows of tiles are
"interleaved" in the sense that the tiles in a row appear each two columns
alternately with the tiles in the adjacent rows, we can represent two rows of
the level in one grid line where the tiles of one row are in the columns with
odd column indexes and the tiles of the other row in the columns with even
column indexes. Naturally, we must then be careful when doing movements around
the grid.

\subsubsection{The main idea of the encoding}
The problem of deciding whether a Hex-a-Hop level can be solved can be
transformed into the parameterized problem of deciding whether a Hex-a-Hop level
can be solved in a given number of steps.
The main idea of the encoding into a SAT CNF formula is to use a set of main
variables indicating the position of the player at each timestep, that is to say
the set of three-dimensional vectors $(x, y, t)$ where $x$ is the abscissa of
the position, $y$ the ordinate of the position and $t$ the discrete timestep.

Obviously, the first set of constraints (clauses) we therefore have to add are
what we've called the {\em state clauses} :
\begin{itemize}
\item At each timestep, the player has to be somewhere, on a tile that is
      accessible (no water), therefore we have the clauses containing all the
      possible positions for a given timestep t :\\
      $\bigwedge \limits_{t=0}^{t_{max}}
       (\bigvee \limits_{\substack{i \in [\![0, x_{max}]\!]\\
				   j \in [\![0, y_{max}]\!]\\
				   (x_i, y_i) \, accessible}}
	(x_i, y_j, t))$

\item But the player can't be at two positions at the same time, therefore we
      have all the clauses containing the negation of two different positions
      for a given timestep t :\\
      $\bigwedge \limits_{t=0}^{t_{max}}
       (\bigwedge \limits_{\substack{\{(i, j), (i', j')\} \in
				     \mathcal{P}({[\![0, x_{max}]\!] \times
						  [\![0, y_{max}]\!]})}
				     }
	(\neg(x_i, y_i, t) \vee \neg(x_{i'}, y_{j'}, t)))
       $
\end{itemize}

\subsubsection{The dynamic type of tiles}
In our first version of the encoding, there were no more variables than those
encoding the position at each timestep. But it is convenient to add some other
variables, auxiliary variables, that help to memorize some useful states.
Dynamic type variables therefore memorize the type a tile has at a given
timestep, as it can be green, turquoise... These variables must be initialized
correctly thanks to the initial state in the level, and their value must be
correctly updated. Therefore, for each position and timestep $(x, y, t)$, we
have the following implications (clauses) that cover all the possible cases :\\
$\neg(x, y, t-1, d_{turquoise}) \Rightarrow (x, y, t, d_{turquoise})\\
 (x, y, t-1, d_{turquoise}) \wedge (x, y, t-1) \Rightarrow
					       \neg(x, y, t, d_{turquoise})\\
 (x, y, t-1, d_{turquoise}) \wedge \neg(x, y, t-1) \Rightarrow
						   (x, y, t, d_{turquoise})\\
 \neg(x, y, t-1, d_{green}) \wedge \neg(x, y, t-1, d_{turquoise}) \Rightarrow
						      \neg(x, y, t, d_{green})\\
 (x, y, t-1, d_{turquoise}) \wedge (x, y, t-1) \Rightarrow
					       (x, y, t, d_{green})\\
 (x, y, t-1, d_{turquoise}) \wedge \neg(x, y, t-1) \Rightarrow
						   \neg(x, y, t, d_{green})\\
 (x, y, t-1, d_{green}) \wedge (x, y, t-1) \Rightarrow
					   \neg(x, y, t, d_{green})\\
 (x, y, t-1, d_{green}) \wedge \neg(x, y, t-1) \Rightarrow
					       (x, y, t, d_{green})
$


\section{Conclusion}
To complete.

\end{document}
